\documentclass[12pt, a4paper]{book}
%=========================================CÁC GÓI VÀ HÀM CHUẨN BỊ==========================================
% Cho phép sử dụng màu và vẽ viền
\usepackage{xcolor}
\usepackage{tikz}
\usepackage{contour}
\usepackage{gradient-text}

\definecolor{vol}{HTML}{646464}
\definecolor{eptype}{HTML}{2f5496}      
\definecolor{epnum}{HTML}{4472c4}       
\definecolor{epattr}{HTML}{b4c6e7}

\definecolor{song}{HTML}{7030a0}
\definecolor{ai}{RGB}{91,155,213}
\definecolor{manzai}{HTML}{005900}
\definecolor{bdaya}{HTML}{E31BAF}
\definecolor{bdayb}{HTML}{FFC000}

\definecolor{holofes}{HTML}{ffc000}
\definecolor{holop}{HTML}{9E0000}

% Màu nhân vật
% \definecolor{Suichan}{HTML}{9CD1F8}
\definecolor{Robopon}{RGB}{255,98,98}
\definecolor{Mikopi}{RGB}{254,212,212}
\definecolor{Polmao}{HTML}{fe0f80}
\definecolor{Wamy}{HTML}{a0f9fe}
\definecolor{Shishiron}{HTML}{a2e4ce}

% Công thức toán học
\usepackage{amsmath}
\usepackage{amssymb}
\usepackage{unicode-math}
\usepackage{gensymb}

% Hộp văn bản
\usepackage{tcolorbox}
\usepackage{soul}

% Liên kết
\usepackage{hyperref}
\usepackage{fixfoot}

% Chèn các file nhỏ
\usepackage{subfiles}

% Chèn ảnh, video và GIF
\usepackage{graphicx}
\usepackage{wrapfig}
\input{media_embed.tex}	
\usepackage{animate}

% Lề và cỡ chữ tiêu chuẩn
\usepackage{extsizes}                       % Cho cỡ chữ 14pt
\usepackage[margin=1.54cm]{geometry}        % Căn chỉnh lề giấy
\usepackage{parskip}
\usepackage{emptypage}

% Gạch chân xuống dòng
\usepackage{ulem}

% Thay đổi thiết lập chú thích
\usepackage{caption}
\usepackage{subcaption}

\captionsetup{justification= centering, font=small, labelformat=empty}

% Sử dụng để thay đổi font
\usepackage{fontspec}
% \usepackage{luatexja}
% \usepackage{luatexja-fontspec}

% Viết furigana
\usepackage{ruby}
\renewcommand{\rubysize}{0.4}
\renewcommand{\rubysep}{1pt}

% Phông cho tên-số các tập
\newfontface\flaregothic[Path=Fontface/]{flaregothicbold.ttf}
\newfontface\cafeteria[Path=Fontface/]{Cafeteria Bold.otf}
\newfontface\fotskip[Path=Fontface/]{FOT-Skip Std B.otf}
\newfontface\vnmsans[Path=Fontface/]{VNM Sans Std.otf}
\newfontface\vnmdisp[Path=Fontface/]{VNM Sans Display Bold.otf}
\newfontface\sen[Path=Fontface/]{Sen-Bold.ttf}
\newfontface\continuum[Path=Fontface/]{contb.ttf}
\newfontface\japfont[Path=Fontface/]{msgothic.ttc}
\newfontface\jiga[Path=Fontface/]{Jigmo.ttf}
\newfontface\jigb[Path=Fontface/]{Jigmo3.ttf}
\newfontface\jigc[Path=Fontface/]{Jigmo3.ttf}
\newfontface\quincy[Path=Fontface/]{Quincy Caps Regular.ttf}

% Phông cho tiêu đề cuốn truyện và đầu-chân trang
\newfontface\frank[Path=Fontface/]{Franklin Gothic Heavy Regular.ttf}
\newfontface\caviar[Path=Fontface/]{VNF-Cariar Dreams Regular.ttf}
\newfontface\hbrk[Path=Fontface/]{HouseBrokenRough.ttf}
\newfontface\ab[Path=Fontface/]{angrybirds-regular.ttf}
\newfontface\abv[Path=Fontface/]{0259-LNTH-angrybirds-regular.ttf}
\newfontface\device[Path=Fontface/]{Device Regular.ttf}
\newfontface\harosbaelz[Path=Fontface/]{BAEFONT_NORMAL-REGULAR_V1.TTF}
\newfontface\brian[Path=Fontface/]{Brianne_s_hand.ttf}
\newfontface\nexa[Path=Fontface/]{Nexa-Heavy.ttf}
\newfontface\digi[Path=Fontface/]{DS-DIGI.TTF}
\newfontface\freshbot[Path=Fontface/]{FRESHBOT.TTF}

% Đặt font chính và font tiếng Nhật
\setmainfont{Times New Roman}
\setmathfont
[    Extension = .otf,
     BoldFont = XITSMath-Bold,
]
{XITSMath-Regular}

% Điểu chỉnh đầu trang và chân trang
\usepackage{fancyhdr}
\usepackage{spverbatim}

% Dùng cho điều chỉnh kích thước và vị trí bảng
\usepackage{array}
\usepackage{changepage}

% Nhóm cột
\usepackage{multirow}
\usepackage{float}
\usepackage{multicol}

% Dùng cho đoạn hội thoại
\usepackage{setspace} % Chia khoảng cách
\usepackage{dialogue}
\usepackage{calc} % Thiết lập khoảng cách giữa tên nhân vật và văn bản
\usetikzlibrary{calc}

% Tăng chỉ số chiều sâu của sách
\setcounter{tocdepth}{4}
\setcounter{secnumdepth}{4}

%==================================================MACRO===================================================
% A. Đường thẳng, ký hiệu
\newcommand{\ct}
{\begin{center}[---]\end{center}} 		  	                          % Cắt cảnh dài
\newcommand{\chrint}        
{\begin{center}||---||\end{center}}						        	            % Cắt đến giói thiệu nhân vật, có thể bỏ qua
\newcommand{\pov}[1]
{\begin{center}$\langle${#1}$\rangle$\end{center}}			            % Đổi góc nhìn nhân vật (mặc định là ngôi thứ ba)
\newcommand{\il}{\noindent\makebox[\textwidth]
{\rule{\paperwidth}{0.4pt}}}					  			                      % Đường thẳng phân chia

% B. Chèn ảnh và gif
\newcommand{\img}[2][12.5cm]{%
    \begin{figure}[H]
        \centering
        \includegraphics[width=#1]{Image/#2.png}                    % Tự động thêm "Image/" vào trước tên tệp; chỉ cần chèn tên tệp
    \end{figure}
}

\newcommand{\gif}[4][12.5cm]{%
    \begin{figure}[H]
        \centering
        \animategraphics[autoplay,loop,width=#1]
        {30}{Image/#2/#2.}{#3}{#4}                                  % Chỉ cần chèn tên tệp, lấy từ khung #3 tới #4
    \end{figure}
}

\newcommand{\pictdum}{										                          % Ảnh chèn vào thay thế ảnh gif để canh dòng
        \includegraphics[width=8cm]{placeholder.png}
}
    
\newcommand{\vid}[3][12.5cm]{%
  \embedvideo{\includegraphics[width=#1]{Image/#2.png}}{#3.mp4}
}

% C. Viết cụm và định dạng
\newcommand{\omk}{\begin{center}{\abv\Large Truyện phụ}\end{center}}
\newcommand{\anni}{\color{bdaya}A\color{bdayb}N\color{bdaya}N\color{bdayb}I\color{bdaya}V\color{bdayb}E\color{bdaya}R\color{bdayb}S\color{bdaya}A\color{bdayb}I\color{bdaya}R\color{bdayb}E}
\newcommand{\xmas}{\color{green}N\color{gray}O\color{red}Ë\color{green}L}
\newcommand{\hlw}{\color{purple}H\color{brown}A\color{purple}L\color{brown}L\color{purple}O\color{brown}W\color{purple}E\color{brown}E\color{purple}N}
\newcommand{\trva}{\begin{center}{\abv\Large Bạn có biết?}\end{center}}
\newcommand{\hll}{\textit{hololive}}
\newcommand{\stpt}[1]{\begin{center}{\sen\Large {#1}} \end{center}}     % Subsection
\newcommand{\stcap}[1]{\begin{center}{\caviar {#1}} \end{center}}       % Phụ đề cho subsection
\newcommand{\xdo}{\\[-24pt]}                                            % Dùng cho phần dialogue

% D. Bảng xếp hàng (cho các tập Top 10)
\newcommand{\topten}{\ruby{大}{だい}\ruby{惨}{さん}\ruby{事}{じ}}
\newcommand{\bxh}[4]{
    \begin{center}
        {{\cafeteria\Large Number #1:} \\
        {\textbf{#2}} \\
        {(Từ {\flaregothic\textbf{ÉPISODE #3}}, quyển số #4.)}}
    \end{center}
}

% E. Ngôn ngữ
\newcommand{\vie}[1]{\color{red} #1 \color{black}}        % Tiếng Việt (Etsunan ở Round 1 và 2)
\newcommand{\eng}[1]{\color{blue} #1 \color{black}}       % Tiếng Anh
\newcommand{\ind}[1]{\color{green} #1 \color{black}}      % Tiếng Indonesia
\newcommand{\chn}[1]{\color{orange} #1 \color{black}}     % Tiếng Trung

% Thiết đặt môi trường cho đoạn hội thoại
% A. Dùng để viết hoa chữ đầu tiên tên nhân vật
\renewcommand*\DialogueLabel[1]{%
  {#1}:%
}

% B. Thiết lập với độ dài tên dài nhất. Mặc định chọn Hanamaru từ Love Live!.
\newlength{\widestname}
\setlength{\widestname}{2.5cm}

% C. Tái định nghĩa lại hàm dialogue
\makeatletter
\renewenvironment{dialogue} {%
    \begin{list}{} {%
        \setlength\itemsep{\z@ \@plus .5ex}%
        \setlength{\parsep}{\parskip}%
        \setlength{\rightmargin}{0pt}           % Không căn lề phải
        \setlength{\labelwidth}{\widestname}    % Thiết đặt khoảng cách với tên dài nhất
        \setlength{\labelsep}{0.5em}            % Thiết đặt khoảng cách giữa tên nhân vật và đoạn hội thoại
        \setlength{\leftmargin}{\labelwidth}    % Căn lề trái
        \addtolength{\leftmargin}{\labelsep}    % 
        \defcommand\speak [1] {\item[{##1}]}    % Thiết lập lệnh speak trong dialogue
        \let\makelabel\DialogueLabel
      }%
      \PreDialogue\relax
    }{%
  \end{list}%
  }
\makeatother

% D. Hướng mũi tên
\newcommand{\arrow}[1][-45]
      {\begin{tikzpicture}[baseline = -0.5ex]
          \node[inner sep=0pt,outer sep=0pt,rotate = #1] (a) at (0,0)  {\rightarrow};
       \end{tikzpicture}}

% E. Chiêu mộ toàn bộ ký tự
\makeatletter
\font\SOUL@tt="Times New Roman"
\setbox\z@\hbox{\SOUL@tt-}
\SOUL@ttwidth\wd\z@ %reset default width of -
\makeatother

% F. Tabbing
\newcommand\tabs[1][1cm]{\hspace*{#1}}

\DeclareFixedFootnote{\rep}{Công ty Google, 1600 Amphitheatre Parkway, Mountain View, California 94043, Hoa Kỳ.}
\DeclareFixedFootnote{\reps}{Trung tâm Kiểm soát và Phòng ngừa dịch bệnh Hoa Kỳ, 1600 Đường Clifton, NE, Atlanta, Georgia 30333, Hoa Kỳ.}

%================ĐẦU TRANG VÀ CHÂN TRANG====================
\pagestyle{fancy}
\fancyhead{} 
% Tên cuốn truyện
\fancyhead[LO]
{\vnmsans\fontsize{12pt}{12pt}\selectfont
\color{vol}{Khai phá dữ liệu}
\\[-12pt]}                      
\fancyhead[RE]
{\vnmsans\fontsize{12pt}{12pt}\selectfont
\color{vol}{Khai phá dữ liệu}
\\[-12pt]}                     
% Số chương
% \fancyhead[RO]
% {\harosbaelz\fontsize{18pt}{18pt}\selectfont
% \color{vol}{seven}
% \ab\fontsize{18pt}{18pt}\selectfont
% \color{vol}{\textbf{(VII)}}
% \\[-18pt]
% }
% \fancyhead[LE]
% {\harosbaelz\fontsize{18pt}{18pt}\selectfont
% \color{vol}{\textbf{sept}}
% \ab\fontsize{18pt}{18pt}\selectfont
% \color{vol}{\textbf{(7)}}
% \\[-18pt]}
% Chân trang
% \fancyfoot{} 
% \fancyfoot[CO, CE]{\thepage}
% \fancyfoot[LO]
% {\ab\fontsize{10pt}{10pt}\selectfont \color{vol} [DIGITAL FORMAT ONLY ⇒ DO NOT PRINT]}
% \fancyfoot[RE]
% {\ab\fontsize{10pt}{10pt}\selectfont \color{vol} [FORMAT NUMÉRIQUE UNIQUEMENT ⇒ NE PAS IMPRIMER]}
%==========================================PHẦN THÂN==========================================
\begin{document}
% \subfile{Cover} % Phần bìa

% \renewcommand{\contentsname}
% {\centering\vnmsans\Large\textcolor{vol}{Mục lục}}
% {
%   \hypersetup{linkcolor=black}
%   \tableofcontents
% }
%========================================LỜI NÓI ĐẦU==========================================
\newpage
% \section*{\phantomsection\label{preface}} 
% \addcontentsline{toc}{section}{\textbf{Lời nói đầu}} 
% {\centering\vnmsans\Large\textcolor{vol}{Lời nói đầu}\par}
% \subfile{intro}

\begin{center}
  {\huge \textbf{Phát hiện dịch cúm bằng cách sử dụng dữ liệu truy vấn của công cụ tìm kiếm}}\\
  bởi Jeremt Ginseberg\rep{}, Matthrw H. Mohebbi\rep{}, Rajan S. Patel\rep{}, Lynnette Brammer\reps{}, Mark S. Smolinski\rep{} \& Larry Brilliant\rep{}\\
  \underline{Đăng ngày 19 tháng 02 năm 2009}
\end{center}

\textbf{Các trận dịch cúm mùa là một mối lo ngại lớn về sức khỏe cộng đồng, gây ra hàng chục triệu ca bệnh đường hô hấp và gây ra từ 250.000 đến 500.000 ca tử vong trên toàn thế giới mỗi năm\textsuperscript{1}. Bên cạnh cúm mùa, một chủng virus cúm mới mà con người chưa có miễn dịch trước đó và có khả năng lây truyền từ người sang người có thể dẫn đến đại dịch với hàng triệu người tử vong\textsuperscript{2}.

Việc phát hiện sớm hoạt động của bệnh, đi cùng với việc phản ứng nhanh chóng, có thể làm giảm tác động của cả cúm mùa và cúm đại dịch\textsuperscript{3,4}. Một cách để cải thiện việc phát hiện sớm là theo dõi hành vi tìm kiếm sức khỏe dưới dạng các truy vấn (queries) đến các công cụ tìm kiếm trực tuyến, vốn được gửi bởi hàng triệu người dùng trên khắp thế giới mỗi ngày.

Ở đây, chúng tôi trình bày một phương pháp phân tích số lượng lớn các truy vấn tìm kiếm của Google để theo dõi bệnh giống cúm (influenza-like illness) trong một quần thể. Do tần suất tương đối của một số truy vấn nhất định có mối liên quan rất lớn với tỷ lệ các lượt khám bác sĩ mà bệnh nhân có triệu chứng giống cúm, chúng tôi có thể ước tính chính xác mức độ hoạt động của cúm hàng tuần hiện tại ở mỗi khu vực của Hoa Kỳ, với độ trễ báo cáo khoảng một ngày. Phương pháp này có thể giúp sử dụng các truy vấn tìm kiếm để phát hiện các vụ dịch cúm ở những khu vực có số lượng lớn người dùng tìm kiếm trên web.}

Các hệ thống giám sát truyền thống, bao gồm cả những hệ thống được sử dụng bởi Trung tâm Kiểm soát và Phòng ngừa Dịch bệnh Hoa Kỳ (CDC) và Chương trình Giám sát Cúm Châu Âu (EISS), dựa trên cả dữ liệu virus học và dữ liệu lâm sàng, bao gồm số lượt khám bác sĩ do bệnh giống cúm (ILI). CDC công bố dữ liệu quốc gia và khu vực từ các hệ thống giám sát này hàng tuần, thường với độ trễ báo cáo từ 1-2 tuần.

Trong nỗ lực cung cấp khả năng phát hiện nhanh hơn, các hệ thống giám sát cải tiến đã được tạo ra để theo dõi các tín hiệu gián tiếp của hoạt động cúm, chẳng hạn như lượng cuộc gọi đến các đường dây tư vấn y tế qua điện thoại\textsuperscript{5} và doanh số bán thuốc không kê đơn\textsuperscript{6}. Khoảng 90 triệu người trưởng thành Mỹ được cho là tìm kiếm thông tin trực tuyến về các bệnh cụ thể hoặc các vấn đề y tế mỗi năm\textsuperscript{7}, khiến các truy vấn tìm kiếm trên web trở thành một nguồn thông tin đặc biệt giá trị về xu hướng sức khỏe. Các nỗ lực trước đây trong việc sử dụng hoạt động trực tuyến để giám sát cúm đã đếm số lượng truy vấn tìm kiếm được gửi đến một trang web y tế của Thụy Điển (A. Hulth, G. Rydevik và A. Linde, bản thảo đang chuẩn bị), số lượt truy cập vào các trang nhất định trên một trang web y tế của Hoa Kỳ\textsuperscript{8} và số lượt nhấp của người dùng vào một quảng cáo từ khóa tìm kiếm ở Canada\textsuperscript{9}. Một tập hợp các truy vấn tìm kiếm của Yahoo có chứa các từ `flu' (cúm) hoặc `influenza' (từ đồng nghĩa với từ trước - cúm) đã được phát hiện là có tương quan với dữ liệu giám sát virus học và tỷ lệ tử vong trong nhiều năm\textsuperscript{10}.

Hệ thống được đề xuất của chúng tôi xây dựng dựa trên công trình trước đó bằng cách sử dụng một phương pháp tự động để khám phá các truy vấn tìm kiếm liên quan đến cúm. Bằng cách xử lý hàng trăm tỷ lượt tìm kiếm riêng lẻ từ nhật ký tìm kiếm trên web của Google trong 5 năm, hệ thống của chúng tôi tạo ra các mô hình toàn diện hơn để sử dụng trong giám sát cúm, với các ước tính về hoạt động ILI ở cấp khu vực và tiểu bang ở Hoa Kỳ. Việc sử dụng rộng rãi các công cụ tìm kiếm trực tuyến trên toàn cầu cuối cùng có thể cho phép phát triển các mô hình trong bối cảnh quốc tế.

%=======================================DANH SÁCH TẬP========================================
% \newpage
% \section*{\phantomsection\label{ep7.1}} 
% \addcontentsline{toc}{section}{\textbf{Đôi nét về phân loại ảnh}} 
% \subfile{Episode/Introduction}

% \newpage
% \section*{\phantomsection\label{ep7.2}} 
% \addcontentsline{toc}{section}{\textbf{Khái niệm}} 
% \subfile{Episode/Def}

% \newpage
% \section*{\phantomsection\label{ep7.3}} 
% \addcontentsline{toc}{section}{\textbf{Thực hành}}
% \subfile{Episode/Code}

% \newpage
% \section*{\phantomsection\label{ep7.3}} 
% \addcontentsline{toc}{section}{\textbf{Tài liệu tham khảo}}
% \subfile{Episode/Ref}

%===========================================LỜI KẾT===========================================
\newpage
% \section*{\phantomsection\label{end}} 
% \addcontentsline{toc}{section}{\textbf{Lời kết}}
% {\centering\vnmsans\Large\textcolor{vol}{Lời kết}\par}
% \subfile{end}

\end{document}